% !TEX root = ../TAMU_Thesis_Main.tex

%%%%%%%%%%%%%%%%%%%%%%%%%%%%%%%%%%%%%%%%%%%%%%%%%%%%%%%%%%%%%%%%%%%%%
%%                           ABSTRACT 
%%%%%%%%%%%%%%%%%%%%%%%%%%%%%%%%%%%%%%%%%%%%%%%%%%%%%%%%%%%%%%%%%%%%%

\begin{abstract} %350 word limit!!!
Numerical ocean models are essential tools for understanding the ocean's influence on Earth's climate and weather. A longstanding issue thought to affect model fidelity is numerical mixing, the spurious mixing generated by the discretization of tracer advection. Little is known about numerical mixing in simulations of submesoscale flows, other than it can be larger than physical mixing, the mixing parameterized by turbulence closure schemes. This work adopts a case study approach using realistic and idealized submesoscale eddy-resolving simulations of the Texas-Louisiana (TXLA) shelf in the Gulf of Mexico (GoM) to improve our understanding of numerical salinity mixing. The objectives of this dissertation are to: 1) quantify the sensitivity of numerical and physical mixing to horizontal resolution, 2) determine whether numerical mixing can be quantified offline with salinity variance budgets, 3) characterize where numerical and physical mixing are significant in the water column, and 4) understand how numerical mixing alters the larger-scale ocean circulation and tracer state. 

We use a two-way nested implementation of the TXLA model based on the Regional Ocean Modeling System (ROMS) to address objectives one and two and an idealized model to address objectives three and four. We find that increasing grid resolution decreases the amount of volume-integrated numerical mixing and increases physical mixing. However, this has limited returns because newly resolved processes (e.g., fronts) locally increase numerical mixing. We find that numerical mixing should not be quantified offline because offline methods do not converge to online methods at sub-hourly output frequencies. The simulations suggest numerical mixing dominates physical mixing in frontal zones and within the mixed layer. We test three tracer advection schemes with idealized simulations and find that the scheme with the most numerical mixing suppresses submesoscale instabilities, implying that numerical mixing may continue to degrade model fidelity even at submesoscale resolving resolution. After achieving the primary objectives, we use knowledge gained from the ROMS simulations to assess regionally refined, submesoscale-permitting simulations in the GoM using the Model for Prediction Across Scales-Ocean (MPAS-O). Results suggest MPAS-O's river forcing implementation and stability issues at high vertical resolution require further improvements for modeling coastal submesoscale processes. 
\end{abstract}