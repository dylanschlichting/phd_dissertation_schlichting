% !TEX root = ../TAMU_Thesis_Main.tex

%%%%%%%%%%%%%%%%%%%%%%%%%%%%%%%%%%%%%%%%%%%%%%%%%%%%%%%%%%%%%%%%%%%%%%
%%                           SECTION V
%%%%%%%%%%%%%%%%%%%%%%%%%%%%%%%%%%%%%%%%%%%%%%%%%%%%%%%%%%%%%%%%%%%%%

\chapter{SUMMARY AND CONCLUSIONS \label{cha:Summary}}
This dissertation used a case study approach to explore numerical and physical salinity mixing in realistic and idealized simulations of submesoscale baroclinic instabilities with the Regional Ocean Modeling System (ROMS). The study site is the Mississippi and Atchafalaya (M/A) River plume over the Texas Louisiana (TXLA) shelf in the northern Gulf of Mexico (nGoM). The region is a known hotspot for submesoscales during summer as frontal eddies are squeezed and stretched by lateral buoyancy forcing from regional rivers and the passage of occasional storms. Many of the research questions outlined in Section \ref{sec:diss_obj} are novel because numerical mixing has not been characterized in submesoscale eddy-resolving simulations. Our findings from two studies are summarized below. 

First, we used a non-nested version of the realistic TXLA model to characterize whether numerical mixing can be quantified offline with the residuals of the volume mean salinity variance $s^{\prime^2}$ and salinity squared $s^2$ budgets. The accuracy of offline methods are evaluated with an analogous online method. We found the accuracy of the offline methods to be strongly dependent on the model output frequency. At an output frequency of one hour, the residual of the $s^2$ budget is noisy and can disagree with the online method by over an order of magnitude. For the same output frequency, the $s^{\prime^2}$ budget overestimates the online method by 60\%. We find that offline methods should not be used to quantify numerical mixing because they do not converge to the online method, even as the output frequency is increased to 10 minutes, which is impractical for many long term coastal simulations. 

Second, we used a two-way nested version of the TXLA model to quantify the sensitivity of numerical and physical mixing to horizontal resolution. We found that volume-integrated numerical mixing comprised 57\% of the physical mixing in the non-nested model. Volume-integrated numerical mixing generally decreases and physical mixing increases as grid resolution is increased. We increased grid resolution by a factor of five, causing numerical mixing to decrease by 35\% and physical mixing to increase by 42\%. Numerical mixing is roughly proportional the magnitude of horizontal salinity gradients, which are strongest at ocean fronts. The across-frontal length scales generally decrease as model resolution increases, so the integrated amount of numerical mixing decreases. However, increased resolution can resolve new processes that create new horizontal gradients, or further sharpen existing gradients via frontogenesis. Thus, increasing resolution may only slightly reduce numerical mixing once a model permits submesoscales. Increased resolution causes physical mixing to increase because newly resolved processes create new sources of salinity variance that can be mixed. New processes also increase the vertical velocity variance of the flow, which causes the salinity diffusivity to increase. 

Third, we used an idealized submesoscale eddy-resolving model of the TXLA shelf to prove that numerical mixing dominates physical mixing in frontal zones. Joint probability density functions of the surface normalized frontogenesis function and numerical mixing show that numerical mixing is elevated throughout a front's life cycle. That is, numerical mixing is stronger when fronts are being sharpened by frontogenesis or destroyed by frontolysis compared to fronts that undergo neither process.

Fourth, we varied the tracer advection scheme in an ensemble with the idealized model to investigate the impacts of numerical mixing on the larger scale ocean circulation and tracer state. A key result is that the scheme (HSIMT) with the most numerical mixing can damp the release of available potential energy by partially suppressing baroclinic instabilities. The strongest evidence for this is the ensemble-averaged eddy kinetic energy in the HSIMT runs is 25\% lower than runs in the scheme (MPDATA) with less than half the numerical mixing. The suppression of instabilities can cause biases in alongshore-averaged salinity sections exceeding 0.75 psu in just 30 days. We expect tracer field biases caused by numerical mixing to be much larger in realistic simulations that employ longer run times. These results critically suggest that numerical mixing may continue to affect model fidelity even at submesoscale-resolving resolution. 

After answering the research questions pertaining to numerical mixing, we used the knowledge gained from the TXLA model simulations to present preliminary developments of a submesoscale-permitting regionally refined mesh in the GoM using the Model for Prediction Across Scales Ocean (MPAS-O). MPAS-O is the global ocean component of the Department of Energy's Energy Exascale Earth System Model. This study was the first to assess MPAS-O's representation of submesoscale processes. Our assessment focused on the seasonal variability  of surface submesoscales in GoM and the structure of M/A river plume. We presented simulations with low (GoM3r1) and medium (GoM3r2) resolutions that were run for two years (2000-2001) and compared with the non-nested TXLA model. We were unable to produce a high vertical resolution case because the model experiences stability issues near the M/A discharge points. We plan to fix these stability issues moving forward. 

While both simulations permit submesoscale seasonal variability in the broader GoM, they poorly represent submesoscales over the TXLA shelf and did not produce an eddy field during summer. In addition, both simulations demonstrate major differences in seasonally averaged sea surface salinity relative to the TXLA model. We anticipate improved performance with MPAS-O once the stability issues are fixed and the implementation of river forcing is adjusted. 

The unifying theme of this dissertation is that we studied several factors that affect numerical model fidelity. As George Veronis showed back in the 1970's, just one poorly implemented or untuned subgrid-scale parameterization can cause numerical models to drift away from reality. We primarily focused on numerical mixing with the TXLA model, but also show this with several aspects of model setup during the MPAS-O study. While numerical mixing is not a parameterization, it does partially represent unresolved processes because it is fundamentally modulated by truncation error. We hope that future studies use the knowledge gained here to develop methods that minimize numerical mixing in primitive equation ocean models. 


