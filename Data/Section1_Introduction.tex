% !TEX root = ../TAMU_Thesis_Main.tex

%%%%%%%%%%%%%%%%%%%%%%%%%%%%%%%%%%%%%%%%%%%%%%%%%%%%%%%%%%%%%%%%%%%%%%
%%                           SECTION I
%%%%%%%%%%%%%%%%%%%%%%%%%%%%%%%%%%%%%%%%%%%%%%%%%%%%%%%%%%%%%%%%%%%%%

%\chapter[Put citation here if part of this work is already published]{Introduction and Literature Review}
\chapter{Introduction}

\section{Background and Motivation}
Numerical ocean models are critical tools for understanding how the ocean shapes climate and weather patterns. They face the formidable challenge of representing a turbulent fluid with interacting scales that range from global to smaller than the head of a pin. Consequently, model fidelity is constrained by horizontal and vertical resolution, the quality of surface and lateral forcing, numerical errors that arise from the representation of physical processes with discrete approximations, and the representation of unresolved processes via subgrid-scale (SGS) parameterizations \citep{fox2019challenges}. 

One example of an unresolved process is mixing, which can be defined as the irreversible loss of scalar variance due to turbulent processes. Mixing occurs on the microscale and must be parameterized because numerical ocean models will not resolve microstructure momentum or tracer gradients for the foreseeable future \citep{fox2014principles, Qu_2022_box}. Turbulence closure schemes are used to parameterize SGS mixing by first providing estimates of eddy viscosity and diffusivity coefficients from the closure problem, which results from Reynolds averaging the momentum and tracer equations over the scales of turbulent eddies and assuming the Reynolds stresses are proportional to the mean velocity gradient of the flow \citep{pope2001turbulent}. However, eddy viscosity and diffusivity do not paint a complete picture of mixing, because in flows that are already well mixed, these coefficients may still be large; variance of a scalar field must be destroyed for mixing to occur \citep{Burchard_2008, MacCready_2018, osborn1972oceanic}. For this work, we define physical mixing as the destruction of tracer variance by turbulent diffusion. Our choice to focus on tracer mixing is motivated near the end of this chapter.

The problem of computing mixing in numerical models is compounded by the ocean's enormous aspect ratio, so the viscosity and diffusivity are split into horizontal and vertical components and calculated separately. Mixing in the open ocean occurs predominantly along isopycnals, so it is common to rotate the diffusion tensor to align with isoneutral surfaces \citep{redi1982oceanic}. As pointed out by George Veronis in the 1970's \citep{veronis1975role}, failure to do so results in so called spurious mixing that can cause the meridional overturning circulation to drift away from reality. Thus, a major challenge of developing accurate SGS parameterizations is that they must be grid scale and flow aware \citep{fox2019challenges}. There are several other sources of spurious mixing in modern ocean models that can affect model fidelity \citep{megann2022assessment, Schlichting23}.  

Recent research has focused on examining numerical mixing, the spurious mixing generated by the discretization of tracer advection, or the bulk transport of tracers by currents \citep{Griffies_2000, Burchard_2008}. Numerical mixing has been a longstanding problem in the ocean modeling community for several reasons: 1) it cannot be easily controlled during model configuration, 2) it is grid scale and flow dependent \citep{Burchard_2008}, 3) it depends on the tracer advection scheme \citep{fofonova2021plume, Kalra_2019}, and 4) there is no agreed-upon method for quantifying it, partially due to the plethora of mixing definitions used in the oceanographic community \citep{Klingbeil_2014, MacCready_2018, winters1995available}. The literature suggests the impacts of numerical mixing on the larger-scale ocean circulation and tracer state are somewhat understood on the nonhydrostatic scales where part of the turbulence cascade is resolved \citep[e.g., large eddy simulations, ][]{domaradzki2003effective, thornber2007implicit} and on the mesoscale eddy-permitting or resolving scales of most global ocean models \footnote{This is reviewed in Chapter 2.}. For all scales, numerical mixing can be larger than physical mixing \citep{domaradzki2003effective, Griffies_2000, Holmes_2021, Wang_2021}. 

In the 21st century, significant effort has focused on understanding submesoscale oceanic processes (submesoscales). Curiously, little is known about numerical mixing in simulations of submesoscale flows. The defining dynamical characteristic of submesoscales is an $\mathcal{O}(1)$ Rossby number $Ro \sim U(fL)^{-1}$, where $U$ is a characteristic velocity scale, $f$ is the Coriolis parameter, and $L$ is a characteristic length scale \citep{McWilliams_2016,taylor2023submesoscale}. This can be approximated as the normalized relative vorticity $\zeta f^{-1}= \partial_x v - \partial_y u$, where $v$ and $u$ are the horizontal velocity components. The Rossby number (normalized vorticity) in this context describes the influence of horizontal advective forces relative to rotational forces and is a common tool for visualizing submesoscales, which is shown throughout this dissertation. 

Submesoscales are also defined by an $\mathcal{O}(1)$ Froude number $Fr \sim U(NL)^{-1}$, where $N$ is the buoyancy frequency. The Froude number describes the ratio of advection relative to stratification. Thus, for submesoscales, the influence of advection and stratification are important relative to the less energetic mesoscales such that quasi- and semigeostrophic theory are not always suitable to describe their dynamics. Manifesting as fronts, filaments, and eddies, upper ocean submesoscales are thought to generated by frontogenesis, baroclinic instability, and topographic wakes \citep{mcwilliams2019survey, taylor2023submesoscale}. Additional properties of submesoscales are discussed throughout this disssertation. 

\section{Objectives, approach, and organization} \label{sec:diss_obj}
The overarching objective of this dissertation is to improve our understanding of numerical mixing in submesoscale eddy-resolving simulations. A heuristic hypothesis tested throughout this dissertation is that numerical mixing predominantly occurs at ocean fronts. The heuristic component comes into play because fronts are boundaries between water masses. So, we expect truncation errors in the advection scheme to be large when rapid changes to velocity and tracers are discretized over a small number of a grid cells. There is some evidence for this in high resolution estuarine and coastal models \citep{Broatch_2022, Kalra_2019, Ralston_2017, wang2021structure} and in mesoscale eddy-permitting models \citep{Holmes_2021, megann2022assessment}. However, none of these studies focused on submesoscale eddy-resolving flows. 

Since little is known about numerical mixing at these energetic scales, we seek answers to the following questions to provide foundations for future work: 1) Can numerical mixing be quantified robustly offline with model output?, 2) How sensitive is numerical mixing to horizontal resolution?, 3) Does numerical mixing dominate physical mixing in frontal zones?, and 4) How does numerical mixing affect the larger-scale ocean circulation and tracer state?. These questions are ordered to reflect the knowledge gained chronologically about numerical mixing throughout the research process; this will become clearer below as we outline the dissertation. In addition, these questions are motivated by the push of limited domain regional models and some global models towards submesoscale eddy-permitting and resolving resolutions. Submesoscale eddy-resolving simulations are computationally expensive and generate large output. So, it is helpful to understand whether we can quantify numerical mixing without having to rerun models and understand how numerical mixing affects model fidelity as grid resolution increases.

To achieve the primary objective, test the hypothesis, and answer the research questions, we adopt a case study approach and examine submesoscales over the Texas-Louisiana (TXLA) continental shelf in the northern Gulf of Mexico (nGoM). Each summer, a weakly upcoast diurnal land sea breeze superimposed with rotational forcing from the regional inertial period causes the Mississippi/Atchafalaya (M/A) River plume to generate submesoscale baroclinic instabilities with strong inertial currents \citep{Hetland_2017, Kobashi_2020, qu2022rapid}. The lateral density gradients are largely controlled by salinity variations from the M/A plume, so we focus on numerical and physical salinity mixing. The study region is described in Chapters 2-4. The primary tool used in this work is the realistic, validated, TXLA shelf model \citep{Kobashi_2020, Schlichting23, Zhang_2012_forecast}, which is an implementation of the Regional Ocean Modeling System \citep[ROMS, ][]{shchepetkin2005regional}. Various iterations of the model setup and configuration are described in Chapters 2-4. The rest of this dissertation is organized as follows: 

Chapters 2-3 are focused on the primary objective. Numerical-, spurious-, and physical mixing have multifarious meanings and computational methods within the ocean modeling community. We spend a significant portion of Chapter 2 reviewing the literature, defining terminology, and describing methods. We use non-nested and two-way nested versions of the TXLA model to quantify the sensitivity of numerical and physical mixing to horizontal resolution, examine whether numerical mixing can be robustly quantified offline with salinity variance budgets, and correlate numerical mixing with physical processes. We deal mostly with volume-integrated quantities there, as our focus is getting a sense of scale of the numerical and physical mixing in the model. The advantage of comparing simulations with two different resolutions is that it allows us to build intuition about what types of processes emerge at submesoscale eddy-resolving resolution. This work has been published in the refereed journal \href{https://agupubs.onlinelibrary.wiley.com/doi/epdf/10.1029/2022MS003380}{\textit{JAMES}}. The full citation for the chapter is:

\begin{itemize}
    \item[] \textbf{Schlichting, D.}, Qu, L., Kobashi, D., \& Hetland, R. (2023). Quantification of Physical and \\
    \hspace{\labelwidth}\phantom{\texttt{type}-} Numerical Mixing in a Coastal Ocean Model Using Salinity Variance Budgets.\\ 
        \hspace{\labelwidth}\phantom{\texttt{type}-} \textit{Journal of Advances in Modeling Earth Systems}, 15, e2022MS003380. \\
        \hspace{\labelwidth}\phantom{\texttt{type}-} https://agupubs.onlinelibrary.wiley.com/doi/epdf/10.1029/2022MS003380
\end{itemize}

After learning that numerical mixing is correlated with horizontal salinity gradients in Chapter 2 and this relationship slightly weakens with increasing resolution, Chapter 3 explores whether numerical mixing dominates physical mixing in frontal zones and within the mixed layer. We use idealized simulations of the TXLA shelf based on the model configuration described in \cite{Hetland_2017}. An idealized model allows us to prescribe model forcing and isolate specific processes that may be obfuscated by the interacting, multiscale physics found in the realistic model. Then, we vary the tracer advection scheme to explore the impacts of numerical mixing on the larger-scale ocean circulation and tracer state. We argue that numerical mixing suppresses instabilities and may continue to decrease model fidelity even at submesoscale eddy-resolving resolution. This work is under review for publication in the refereed journal \href{https://essopenarchive.org/users/750365/articles/722044-numerical-mixing-suppresses-submesoscale-baroclinic-instabilities-over-sloping-bathymetry}{\textit{JAMES}}. The full citation for the preprint is as follows:
\begin{itemize}
    \item[] \textbf{Schlichting, D.}, Hetland, R., \& Jones, S. (2024). Numerical mixing suppresses \\
    \hspace{\labelwidth}\phantom{\texttt{type}-} submesoscale baroclinic instabilities over sloping bathymetry. \textit{ESS Open Archive.}\\ 
        \hspace{\labelwidth}\phantom{\texttt{type}-} doi: 10.22541/essoar.170983234.49281676/v1
\end{itemize}

After achieving the primary objective, Chapter 4 uses the knowledge gained from previous studies with the TXLA model to examine key aspects of model configuration required to produce high fidelity simulations of submesoscales in the nGoM and the M/A river plume. We present the development of an unprecedented, submesoscale eddy-permitting mesh regionally refined within the GoM using the Model for Prediction Across Scales-Ocean (MPAS-O), the ocean component of the Energy Exascale Earth System model (E3SM) developed by the U.S. Department of Energy. We qualitatively compare MPAS-O's representation of surface submesoscales with two vertical meshes and the M/A river plume structure relative to the TXLA model. We identify strengths and weaknesses of the MPAS configuration so future simulations of submesoscales in buoyancy driven flow can be better represented. 

Chapter 5 presents a summary and conclusions of the completed work. 

