% !TEX root = ../TAMU_Thesis_Main.tex

%%%%%%%%%%%%%%%%%%%%%%%%%%%%%%%%%%%%%%%%%%%%%%%%%%%%%%%%%%%%%%%%%%%%%%
%%                           SECTION I
%%%%%%%%%%%%%%%%%%%%%%%%%%%%%%%%%%%%%%%%%%%%%%%%%%%%%%%%%%%%%%%%%%%%%

%\chapter[Put citation here if part of this work is already published]{Introduction and Literature Review}
\chapter{Introduction}

\section{Background and Motivation}
Numerical ocean models are critical tools for understanding how the ocean shapes climate and weather patterns. They face the formidable challenge of representing a turbulent fluid with interacting scales that range from global to smaller than the head of a pin. Consequently, model fidelity is constrained by resolution in the horizontal and vertical, numerical errors that arise from representing various physical processes with discrete approximations, and the representation of unresolved processes via subgrid-scale (SGS) parameterizations \citep{fox2019challenges}. 

One example of an unresolved process is mixing, which can be defined as the irreversible loss of scalar variance due to turbulent processes. Mixing occurs on the microscale and must be parameterized because numerical ocean models will not resolve microstructure momentum or tracer gradients for the foreseeable future \citep{fox2014principles, Qu_2022_box}. Turbulence closure schemes are used to parameterize SGS mixing by first providing estimates of eddy viscosity and diffusivity from the closure problem, which arises as a consequence of Reynolds averaging the momentum and tracer equations and assuming the Reynolds stresses are proportional to the mean velocity gradient of the flow \citep{pope2001turbulent}. However, eddy viscosity and diffusivity do not paint a complete picture of mixing, because in flows that are already well mixed, these coefficients may still be large; variance of a scalar field must be destroyed for mixing to occur \citep{MacCready_2018, osborn1972oceanic}. For this work, we define physical tracer mixing as the destruction of tracer variance. 

The problem of computing mixing in numerical models is compounded by the ocean's enormous aspect ratio, so the viscosity and diffusivity are split into horizontal and vertical components and calculated separately. Mixing in the open ocean occurs predominantly along isopycnals, so it is common to rotate the diffusion tensor to align with isoneutral surfaces \citep{redi1982oceanic}. As pointed out by George Veronis in the 1970's \citep{veronis1975role}, failure to do so results in so called spurious mixing that can degrade the larger-scale circulation over entire ocean basins. Thus, a major challenge of developing accurate SGS parameterizations is they must be grid scale and flow aware \citep{fox2019challenges}. There are several other sources of spurious mixing in modern ocean models that can affect model fidelity \citep{megann2022assessment, Schlichting23}.  

Recent research has focused on examining numerical mixing, the spurious mixing generated by the discretization of tracer advection, or the bulk transport of tracers by currents \citep{Griffies_2000, Burchard_2008}. Numerical mixing has been a longstanding problem in the ocean modeling community for several reasons: 1) it cannot be easily controlled during model configuration, 2) it is grid scale and flow dependent \citep{Burchard_2008}, 3) it depends on the tracer advection scheme \citep{fofonova2021plume, Kalra_2019}, and 4) there is no agreed-upon method for quantifying it, partially due to plethora of mixing definitions used in the oceanographic community \citep{Klingbeil_2014, MacCready_2018, winters1995available}. The literature suggests the impacts of numerical mixing on the larger-scale ocean circulation and tracer state are somewhat understood on the nonhydrostatic scales where part of the turbulence cascade is resolved \citep[e.g., large eddy simulations, ][]{domaradzki2003effective, thornber2007implicit} and on the mesoscale eddy-permitting or resolving scales of most global ocean models. For all scales, numerical mixing can be larger than physical mixing \citep{Griffies_2000, Wang_2021}. 

In the 21st century, significant scientific effort has focused on understanding submesoscale processes. Curiously, little is known about numerical mixing in simulations of submesoscale flows. The defining dynamical characteristic of submesoscales is an $\mathcal{O}(1)$ Rossby number $Ro \sim U(fL)^{-1}$, where $U$ is a characteristic velocity scale, $f$ is Coriolis parameter, and $L$ is a characteristic length scale \citep{McWilliams_2016,taylor2023submesoscale}. This can be rewritten as the normalized relative vorticity $\zeta f^{-1}= \partial_x v - \partial_y u$, where $v$ and $u$ are horizontal velocity components. The Rossby number describes the influence of inertial (advective) forces relative to rotational forces. Thus, for submesoscales, the influence of advection is important relative to the less energetic mesoscales and quasigeostrophic theory is not always suitable to describe their dynamics. Manifesting as fronts, filaments, and eddies, upper ocean submesoscales are thought to generated by frontogenesis, baroclinic instability, and topographic wakes \citep{mcwilliams2019survey, taylor2023submesoscale}. Additional properties of submesoscales are discussed throughout this work. 

\section{Objectives, approach, and organization} \label{sec:diss_obj}
The primary objective of this dissertation is to improve our understanding of numerical mixing in submesoscale eddy-resolving simulations. A heuristic hypothesis tested throughout this dissertation is that numerical mixing predominantly occurs at ocean fronts. The heuristic component comes into play because fronts are boundaries between water masses. So, we expect truncation errors in the advection scheme to be larger when flow properties are rapidly changed over a small number of grid cells compared to more homogeneous parts of the water column. There is evidence for this in high resolution estuarine and coastal models \citep{Broatch_2022, Kalra_2019, Ralston_2017, wang2021structure} and in mesoscale eddy permitting models \citep{Holmes_2021, megann2022assessment}. However, none of these studies focused on flows in the submesoscale regime.

Since little is known about numerical mixing at these energetic scales, we seek answers to the following questions that will provide a foundation for future work: 1) Does numerical mixing dominate physical mixing in frontal zones?, 2) How sensitive is numerical mixing to horizontal resolution?, 3) Can numerical mixing be quantified robustly offline with model output?, and 4) How does numerical mixing affect the larger-scale ocean circulation and tracer state?. These questions are motivated by the push of limited domain regional models and global models towards submesoscale permitting- and resolving resolutions. Submesoscale simulations are computationally expensive and generate large output. So, it is helpful to understand whether we can quantify numerical mixing without having to rerun models, and understand how numerical mixing affects model fidelity as a grid resolution increases.

To achieve the primary objective, test the hypothesis, and answer the research questions, we adopt a case study approach and examine submesoscales over the Texas-Louisiana (TXLA) continental shelf in the northern Gulf of Mexico (nGoM). Each summer, a weakly upcoast diurnal land sea breeze superimposed with the regional inertial period causes the Mississippi/Atchafalaya (M/A) River plume to generate submesoscale baroclinic instabilities with strong inertial currents \citep{Hetland_2017, Kobashi_2020, qu2022rapid}. The lateral density gradients during summer are largely controlled by salinity variations due to the plume, so we focus on numerical and physical salinity mixing. The study region is described in each of the main chapters. The primary tool used in this work is the realistic, validated, TXLA shelf model \citep{Kobashi_2020, Schlichting23, Zhang_2012_forecast}, which is an implementation of the Regional Ocean Modeling System \citep[ROMS][]{shchepetkin2005regional}. The model setup is described in each of the main chapters. The rest of this dissertation is organized as follows: 

Chapters two and three are focused on the primary objective. Numerical-, spurious-, and physical mixing have multifarious meanings and computational methods within the ocean modeling community. We spend a significant portion of Chapter two reviewing the literature, defining terminology, and describing methods. We use non-nested and two-way nested versions of the TXLA model to quantify the sensitivity of numerical and physical mixing to horizontal resolution, examine whether numerical mixing can be robustly quantified offline with salinity variance budgets, and correlate numerical mixing with physical processes. We deal mostly with volume-integrated quantities there, as our focus is getting a sense of scale of the numerical and physical mixing in the model. 

After learning that numerical mixing is correlated with horizontal salinity gradients in Chapter two, Chapter three explores whether numerical mixing dominates in frontal zones and within the mixed layer, where horizontal salinity gradients are strongest in the water column. We use idealized simulations of the TXLA shelf based on the model configuration described in \cite{Hetland_2017}. An idealized model allows us to prescribe model forcing and isolate specific processes that are otherwise obfuscated in the realistic model. Then, we vary the tracer advection scheme to explore the impacts of numerical mixing on the larger-scale ocean circulation and tracer state. We argue that numerical mixing suppresses instabilities and may continue to decrease model fidelity even at submesoscale eddy-resolving resolution. 

After achieving the primary objective, Chapter four uses the knowledge gained from previous studies with the TXLA model to examine key aspects of model configuration required to produce high fidelity simulations of submesoscales in the nGoM and the M/A river plume. There, we present the development of an unprecedented, submesoscale-permitting mesh regionally refined within the GoM using the Model for Prediction Across Scales-Ocean (MPAS-O), the ocean component of the Energy Exascale Earth System model (E3SM) developed by the U.S. Department of Energy. We qualitatively compare MPAS-O's representation of surface submesoscales with two vertical meshes and the M/A river plume structure relative to the TXLA model. We identify strengths and weaknesses of the MPAS configuration so future simulations of submesoscales in buoyancy driven flow can be better represented. 

Chapter five presents a summary and conclusions of the completed work. 

